\documentclass[11pt]{article}
\bibliographystyle{siam}
\usepackage[margin=0.75in]{geometry}

\title{Homework 1: Representation}
\author{
  Shishido, Juan\\
  \texttt{juanshishido}
}

\begin{document}
\maketitle

\section{Ideal Features}

\subsection{Movies}

In this subsection, I discuss the ideal features for the Academy Award for Best 
Picture. Of the six categories we are trying to predict, this is the only one
that is specifically about the film and not individual performances.

According to a data visualization by Bloomberg Business\cite{bloomberg}, there
are two features that have, historically, been most predictive for winning
Best Picture. These are genre and release date.

An overwhelming number of winners in this category have been dramas. Bloomberg
lists seven genres: drama, biography, musical, romance, thriller, comedy, and
adventure. Each genre would be represented as a binary feature and each film
could belong to a single genre.

The release date data is quite interesting. Bloomberg groups films by the
quarter of the year in which they were released. They note that a majority of
Oscar-winning films are released in the last quarter---September through
December---around the time that nominations are decided. This data would be
represented as a set of four binary features, each corresponding to a quarter
of the year.

Bloomberg also list a film's budget and box office earnings as important
factors. However, their binning is, in my opinion, quite arbitrary. For budget,
they group films that spent \$4 million or less, between \$4 and \$16 million, 
between \$16 and \$64 million, more than \$64 million, and films where data was
unavailable. I believe that cost and earnings data is important. However, I
would represent it as a continuous variable. Depending on the observed
distribution of the data, it might make sense to transform by, for example,
taking the $log$.

Another interesting set of features are those related to the results of other
awards. FiveThirtyEight's Walt Hickey notes that, their model, relies on data
from awards that historically predict the Oscars\cite{fivethirtyeight}. They
use information on both nominations and winners. Some examples include the
Golden Globes, the Critic's Choice Movie Awards, and the Producers Guild of
America awards ceremony. For each award, there would be two binary features,
one representing whether the film was nominated and another representing
whether the film won.

Another potentially important set of features relate to the invididuals
involved with the particular film. For example, a well-known or well-respected
director may be more likely to produce high-quality films. This does not have
to only include directors who have won the Academy Award for Best Director. In
fact, according to Wikipedia\cite{wikibestdirector}, most directors with more
than one win have only won twice. Studios, similarly, may play a role in
producing Oscar-worthy films. The Bloomberg visualization mentioned previously
notes that Columbia Pictures has the most studio wins. For directors and
studios, the features would be continuous. The value would be determined by the 
total number of awards won, normalized by the total number of films created.
Most of the time, this value would be between $0$ and $1$, but it is possible,
for some directors or studios, that this number be higher.

Movie ratings are also potentially important. This feature would most likely be
a value between $0$ and $5$. It would be an average of several ratings. The
challenge in this case would be to make sure that every film in the data set
is represented by all of the rating institutions. Otherwise, the numbers may be
skewed if some films are only rated by institutions who, on average, rate
higher than others. Time would be an important consideration here, too. The way
critics or individuals think about a "4-star" rating, for example, may change
over time.

\subsection{People}

For "people-based" awards, Bloomberg again identifies important features. They
distinguish between male and female awards. The most historically predictive
feature, for either sex, is race. According to their data source, only one
non-white female and only seven non-white males have won Best Actress and Best
Actor, respectively. This data might be represented as two binary
features---white and non-white.

\section{Subset to Instantiate}

\subsection{Movies}

\subsection{People}

\bibliography{hw1_shishido_partI} 

\end{document}
