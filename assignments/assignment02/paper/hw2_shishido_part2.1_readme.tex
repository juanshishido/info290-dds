\documentclass[11pt]{article}
\usepackage[margin=1in]{geometry}
\usepackage{csquotes}
\usepackage{listings}
\lstset{basicstyle=\ttfamily\footnotesize,breaklines=true}
\usepackage{hyperref}

\title{Homework 2: Validity}
\author{
  Shishido, Juan\\
  \texttt{juanshishido}
}

\begin{document}
\maketitle

\section{Implementation}

In this part of the assignment, I evaluate the following hypothesis:

\begin{displayquote}
Movies that contain John Goodman have a significantly different number of box
office hits (either higher or lower) than those that do not.
\end{displayquote}

To do this, I implement a permutation test in Python. The null hypothesis, in
this case, is that there is no difference in the proportion of box office hits
between movies with and without John Goodman. The significance level needed in
order to reject the null hypothesis is $\alpha = 0.01$. The effect size and
direction are also discussed.

To create the data set with which the hypothesis would be evaluated, I merged
the \lstinline{movie.feature.txt} and \lstinline{movie.box_office.txt} files.
The former contained data on 8,304 unique movies. With the latter, I identified
59 movies featuring John Goodman. Of those, 24 (40.7\%) were labeled as box
office hits. Of the 8,245 films not featuring John Goodman, only 2,054 (24.9\%)
were labeled as box office hits.

A permutation test estimates the probability of seeing a test statistic as
extreme as the one observed. This is derived by randomly shuffling the labels,
calculating the test statistic for each of these permutations, and counting the
number of times the permutation-based test statistics were more extreme than
the actual. The resulting proportion is the $\hat{p}$-value.

In my implementation, the labels were randomly shuffled 10,000 times. The
resulting $\hat{p}$-value was 0.0068. Based on this, we reject the null
hypothesis that there is no difference in the proportion of box office hits
between movies with and without John Goodman. The "John Goodman effect" is
positive---that is, movies featuring John Goodman tend to be box office hits
more often than movies without him. Using the information mentioned above, we
can say that the effect size is 15.8 percentage points (40.7\% - 24.9\%).

The Python code used for this analysis can be found on GitHub
\href{https://github.com/juanshishido/info290-dds/blob/master/assignments/assignment02/code/permutation.py}{[link]}.

\end{document}
