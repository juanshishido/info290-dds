\documentclass[11pt]{article}
\usepackage[margin=1in]{geometry}
\usepackage{csquotes}
\usepackage{listings}
\lstset{basicstyle=\ttfamily\footnotesize,breaklines=true}

\title{Homework 2: Validity}
\author{
  Shishido, Juan\\
  \texttt{juanshishido}
}

\begin{document}
\maketitle

\section{On Experiments}

In this part of the assignment, I assess the ways in which Kohavi et al.
establish the nine types of validity outlined by Krippendorff.

Krippendorff describes validity as the collective factors that contribute to
the acceptance of claims.

Kohavi et al. seek to provide a ``practical guide'' for conducting experiments
in an online setting. The authors explain why experimentation is important,
provide several examples, and discuss implementation and evaluation.

Claims that are ``obvious'' or ``make sense'' are said be have face validity.
Krippendorff explains that these type of claims are usually accepted without
requiring ``detailed reasons.'' This does not imply that these type of claims
are infallible, though. In addition face validity, as Krippendorff explains,
is, essentially, an individual perspective. Kohavi et al. claim that,
``Controlled experiments provide a methodology to reliably evaluate ideas.'' To
individuals familiar with research or statistics, this claim is not
unreasonable. That is, it ``makes sense,'' at least based on my experiences and
educational context. Thus, we can say that face validity has been established.

Social validity relates to societal relevance and the way that associated
claims contribtue to public discourse. Kohavi et al. are not explicit about
the social importance of their claim. The examples they provide, with the
exception of the scurvy anecdote in their introduction, all relate to web
products and resources. In this case, social validity was not established.

The remaining types of validity all deal with empirical validity, which relate
to the evidence and theoretical robustness of claims.

Sampling validity relates to the ``degree to which a population is accurately
represented in [a] sample.'' This includes both the way that the sample
represents the population from which it is drawn (``of members'') as well as
the way that it represents \emph{other} populations (``of representatives'').
While Kohavi et al.'s work relates to experiments in an online setting (the
``members'' component), they also provide extensive information that relates to
experiments in general (the ``representatives'' component) and even synthesize
the terminology used across disciplines.

The presented terminology establishes semantic validity, which corresponds to
the contextual accuracy of used terms (or categories). They list and describe
all the components related to experiments.

Kohavi et al. also do well on structural validity. This deals with how well the
``analytical constructs'' represent established practices. The authors provide
information on experimental norms---how to determine and measure outcomes, how
to define experimental units, how to deal with complications, etc.

The processes that Kohavi et al. describe, no doubt, succeed in terms of
functional validity, which deals with use. The experimental method they
describe is widely used in both academia and industry.

They also do well in convergent and discriminant validity. Experiments seek to
control other factors that may influence outcomes of interest.

Experiments, so long as sampling validity is preserved, often yield causal
relationships. This is often the best way to establish predictive validity. Of
course, exogenous factors can always alter those expected outcomes.

\end{document}
